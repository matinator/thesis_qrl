\begin{center}
	\section*{Abstract}
\end{center}
Reinforcement learning is one of three major techniques that allows a model to learn, especially this technique focus on how to create an optimal agent able to reach an objective by interacting with an environment. This thesis tries to analyze the possible potentials and advantages that would derive by using quantum circuits with neural networks; analyzing and explaining how it is possible to create hybrid algorithms that exploit the advantages of both the classical and quantum algorithm.\\
The cartpole environment is tested using the Deep Q-Network and confronting it with the quantum version of it to see if an advantage is present and calculating how much it is compared to the "classical" version.\\
After demonstrating the quantum advantage on the cartpole a more complex environment and with industrial application which is the robotic arm is tested using another kind of algorithm called Soft-Actor Critic. Differently from the DQN version it requires multiple components so different kind of test are runned, such as one component has the quantum variation and another where all of them are.\\
In the end this runs are confronted with all "classical" components with increasing size and compared to the quantum, a clear advantage can be extrapolated demonstrating even future possible application on the industry.

