\begin{center}
	\section*{Abstract}
\end{center}
Reinforcement learning is one of three major techniques that allows a model to learn, especially this technique focus on how to create an optimal agent able to reach an objective by interacting with an environment. 
This thesis tries to analyze the possible potentials and advantages that would derive by using quantum circuits with neural networks; 
analyzing and explaining how it is possible to create hybrid algorithms that exploit the advantages of both the classical and quantum algorithm, expanding the applicability not only on simulators, but even on quantum devices such as Rigetti and IonQ's processors through the amazon AWS braket service. 
Lastly it is analyzed how the quantum noise influence an optimal model obtained through a simulator, by testing it on a quantum processor to understand if it is possible to use the hybrid algorithm on the actual devices.
\newline
\begin{center}
	\printglossary
\end{center}
