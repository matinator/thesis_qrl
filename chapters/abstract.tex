\begin{center}
	\section*{Abstract}
\end{center}
Reinforcement learning is one of three main techniques that allows a model to learn, notably focusing on creating an optimal agent able to reach an objective by interacting with an environment. This thesis tries to analyze the possible potentials and advantages that would derive from using quantum circuits with neural networks. To examine and explain how it is possible to create hybrid algorithms that exploit the improvements of the classical and quantum algorithms.\\
The Cartpole environment tested uses the Deep Q-Network algorithm and is compared with the quantum version to see what kind of advantage is present.
After demonstrating the quantum advantage on the Cartpole, a more complex environment with an industrial application, the robotic arm, is tested using another kind of algorithm called Soft-Actor Critic. Differently from the DQN version, it requires multiple components with different purposes increasing the possible configurations that needs to be tested.\\
For this reason, multiple configurations were run, such as one where only a single component has the quantum variation and one with all components quantum variated.Finally, confronting the quantum variation and all "classical" models, a clear advantage can be extrapolated, showing possible future applications in the industry and other fields.

